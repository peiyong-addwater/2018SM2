\documentclass[paper=a4, fontsize=10pt]{scrartcl} % A4 paper and 10pt font size
        \usepackage{setspace}
        %\onehalfspacing
        \linespread{1.05}% Change line spacing here, Palatino benefits from a slight increase by default
        \usepackage{amsmath,amsfonts,graphicx}
        \usepackage{apacite}
        \usepackage{booktabs}
        %\usepackage{geometry}
        \usepackage[top=1.2in, left=3.8cm, bottom=1.2in, right=3.8cm]{geometry}
        %\usepackage{xcolor}
        \usepackage{graphicx}
        %\usepackage{lipsum}% Used for dummy text.
        \usepackage{wrapfig} % Allows in-line images
        \usepackage{listings}
        \usepackage{color}
        \usepackage{xcolor}
        \definecolor{dkgreen}{rgb}{0,0.6,0}
        \definecolor{gray}{rgb}{0.5,0.5,0.5}
        \definecolor{mauve}{rgb}{0.58,0,0.82}
        \definecolor{titlepagecolor}{cmyk}{1,.60,0,.40}
        \definecolor{namecolor}{cmyk}{1,.50,0,.10} 
        \lstset{frame=tb,
             %language=Java,
             aboveskip=3mm,
             belowskip=3mm,
             showstringspaces=false,
             columns=flexible,
             basicstyle = \ttfamily\small,
             numbers=none,
             numberstyle=\tiny\color{gray},
             keywordstyle=\color{blue},
             commentstyle=\color{dkgreen},
             stringstyle=\color{mauve},
             breaklines=true,
             breakatwhitespace=true,
             tabsize=3
        }
        \usepackage{mathpazo}
        \usepackage[T1]{fontenc} % Use 8-bit encoding that has 256 glyphs
        \usepackage{fourier} % Use the Adobe Utopia font for the document - comment this line to return to the LaTeX default
        \usepackage[english]{babel} % English language/hyphenation
        \usepackage{amsmath,amsfonts,amsthm} % Math packages
        %\usepackage[a4paper, left=2.5cm, right=3cm, top=2.5cm, bottom=2.5cm]{geometry}
        %\usepackage{lipsum} % Used for inserting dummy 'Lorem ipsum' text into the template
        \usepackage{mathpazo} % Use the Palatino font
        \usepackage[T1]{fontenc} % Required for accented characters
        \usepackage{sectsty} % Allows customizing section commands
        \allsectionsfont{\centering \normalfont}%\scshape} % Make all sections centered, the default font and small caps
        
        \usepackage{fancyhdr} % Custom headers and footers
        \pagestyle{fancyplain} % Makes all pages in the document conform to the custom headers and footers
        \fancyhead[L]{\textsc{COMP90007 }}
            \fancyhead[R]{\textsc{Username:} peiyongw}% No page header - if you want one, create it in the same way as the footers below
        \fancyfoot[L]{} % Empty left footer
        \fancyfoot[C]{} % Empty center footer
        \fancyfoot[R]{\thepage} % Page numbering for right footer
        \renewcommand{\headrulewidth}{0pt} % Remove header underlines
        \renewcommand{\footrulewidth}{0pt} % Remove footer underlines
        \setlength{\headheight}{13.6pt} % Customize the height of the header
        
        \numberwithin{equation}{section} % Number equations within sections (i.e. 1.1, 1.2, 2.1, 2.2 instead of 1, 2, 3, 4)
        \numberwithin{figure}{section} % Number figures within sections (i.e. 1.1, 1.2, 2.1, 2.2 instead of 1, 2, 3, 4)
        \numberwithin{table}{section} % Number tables within sections (i.e. 1.1, 1.2, 2.1, 2.2 instead of 1, 2, 3, 4)
        
       % \setlength\parindent{0pt} % Removes all indentation from paragraphs - comment this line for an assignment with lots of text
        
        %----------------------------------------------------------------------------------------
        %	TITLE SECTION
        %----------------------------------------------------------------------------------------
        \makeatletter
        %\renewcommand\@biblabel[1]{\textbf{#1.}} % Change the square brackets for each bibliography item from '[1]' to '1.'
        %\renewcommand{\@listI}{\itemsep=0pt} % Reduce the space between items in the itemize and enumerate environments and the bibliography

        \renewcommand{\maketitle}{ % Customize the title - do not edit title and author name here, see the TITLE block below
        \begin{flushright} % Right align
        {\LARGE\@title} % Increase the font size of the title

        \vspace{50pt} % Some vertical space between the title and author name

        {\large\@author} % Author name
        \\\@date % Date

        \vspace{40pt} % Some vertical space between the author block and abstract
        \end{flushright}
        }

        \newcommand{\horrule}[1]{\rule{\linewidth}{#1}} % Create horizontal rule command with 1 argument of height
        \title{\textbf{Unnecessarily Long Essay Title}\\ % Title
        Focused and Deliciously Witty Subtitle} % Subtitle

        \author{\textsc{Ford Prefect} % Author
        \\{\textit{Interstellar University}}} % Institution

        \date{\today} % Date
        
\begin{document}
        %\thispagestyle{empty}
        \begin{titlepage}
            %\thispagestyle{empty}
            \newgeometry{left=7.5cm} %defines the geometry for the titlepage
            \pagecolor{titlepagecolor}
            \noindent
            %\includegraphics[width=2cm]{logo.jpg}\\[-1em]
            \color{white}
            \makebox[0pt][l]{\rule{1.3\textwidth}{1pt}}
            \par
            \noindent
            \textbf{\textsf{The University of }} \textcolor{namecolor}{\textsf{Melbourne}}
            \vfill
            \noindent
            {\textsf{COMP90007 Internet Technologies SM2, 2018}}
            \vskip\baselineskip
            \noindent
            {\huge \textsf{Research Project}}
            \vskip\baselineskip
            \noindent
            \textsf{Peiyong Wang}
            \vskip\baselineskip
            \noindent
            \textcolor{namecolor}{\textsf{username }}\textsf{peiyongw}
        \end{titlepage}
        \restoregeometry % restores the geometry
        \nopagecolor% Use this to restore the color pages to white
        %\include{titlepage}
        %\setcounter{page}{0}
        %\thispagestyle{empty}
        %\newpage
        \bibliographystyle{apacite}
        %\title{	
       % \normalfont \normalsize 
       % \textsc{The University of Melbourne } \\ [25pt] % Your university, school and/or department name(s)
        %\horrule{0.5pt} \\[0.4cm] % Thin top horizontal rule
       %\huge COMP90007 Internet Technologies SM2, 2018
       % Network Analysis Assignment \\ % The assignment title
       % \horrule{2pt} \\[0.5cm] % Thick bottom horizontal rule
        %}
        
       % \author{Peiyong Wang   955986} % Your name
        
        \maketitle
        \pagenumbering{arabic} % Print the title
        %\thispagestyle{empty}
        %----------------------------------------------------------------------------------------
        %	PROBLEM 1
        %----------------------------------------------------------------------------------------
        \begin{abstract}
                Morbi tempor congue porta. Proin semper, leo vitae faucibus dictum, metus mauris lacinia lorem, ac congue leo felis eu turpis. Sed nec nunc pellentesque, gravida eros at, porttitor ipsum. Praesent consequat urna a lacus lobortis ultrices eget ac metus. In tempus hendrerit rhoncus. Mauris dignissim turpis id sollicitudin lacinia. Praesent libero tellus, fringilla nec ullamcorper at, ultrices id nulla. Phasellus placerat a tellus a malesuada.
        \end{abstract}
                
        \hspace*{3,6mm}\textit{Keywords:} lorem , ipsum , dolor , sit amet , lectus % Keywords
                
        \vspace{30pt} 

        \section{Measuring the hop count}
        This statement requires citation \cite{Smith:2012qr}; this one does too \cite{Smith:2013jd}. Lorem ipsum dolor sit\cite{Chen:2004:UBP:2285778.2286085} amet, consectetur adipiscing elit. Aenean dictum lacus sem, ut varius ante dignissim ac. Sed a mi quis lectus feugiat aliquam. Nunc sed vulputate velit. Sed commodo metus vel felis semper, quis rutrum odio vulputate. Donec a elit porttitor, facilisis nisl sit amet, dignissim arcu. Vivamus accumsan pellentesque nulla at euismod. Duis porta rutrum sem, eu facilisis mi varius sed. Suspendisse potenti. Mauris rhoncus neque nisi, ut laoreet augue pretium luctus. Vestibulum sit amet luctus sem, luctus ultrices leo. Aenean vitae sem leo.

        Nullam semper quam at ante convallis posuere. Ut faucibus tellus ac massa luctus consectetur. Nulla pellentesque tortor et aliquam vehicula. Maecenas imperdiet euismod enim ut pharetra. Suspendisse pulvinar sapien vitae placerat pellentesque. Nulla facilisi. Aenean vitae nunc venenatis, vehicula neque in, congue ligula.

        Pellentesque quis neque fringilla, varius ligula quis, malesuada dolor. Aenean malesuada urna porta, condimentum nisl sed, scelerisque nisi. Suspendisse ac orci quis massa porta dignissim. Morbi sollicitudin, felis eget tristique laoreet, ante lacus pretium lacus, nec ornare sem lorem a velit. Pellentesque eu erat congue, ullamcorper ante ut, tristique turpis. Nam sodales mi sed nisl tincidunt vestibulum. Interdum et malesuada fames ac ante ipsum primis in faucibus.





        \bibliography{myref.bib}
\end{document}



%\begin{align} 
%\begin{split}
%(x+y)^3 	&= (x+y)^2(x+y)\\
%&=(x^2+2xy+y^2)(x+y)\\
%&=(x^3+2x^2y+xy^2) + (x^2y+2xy^2+y^3)\\
%&=x^3+3x^2y+3xy^2+y^3
%\end{split}					
%\end{align}

%\begin{table}[htbp]
 %   \centering
  %  \caption{Expected delay and outgoing line of C}
   % \begin{tabular}{ccc}
   % \hline
   % To &Delay & Line \\
   % \hline
   % A&10&B\\
   % B&5&B\\
   % C&0&-\\
   % D&4&D\\
   % E&3&E\\
   % F&8&D\\
   % \hline
   % \end{tabular}
%\end{table}



%\begin{tabbing}
%\hspace*{.25in} \= \hspace*{.25in} \= \hspace*{.25in} \= \hspace*{.25in} \= \hspace*{.25in} \=\kill
%\>$Euclid(m,n)=$ \\
%\>\> {\bf while} n$ \neq $ 0 \\
%\>\>\> r $ \leftarrow $ $m$ mod $n$  \\
%\>\>\>  m $\leftarrow$n\\
%\>\>{\bf return} m 
%\end{tabbing}

%Python code:
%\begin{lstlisting}[language = python]
%def gcd(m,n):
%	while n != 0:
%		r = m % n
%		m = n
%		n = r
%	return m
%\end{lstlisting}


%\paragraph{Heading on level 4 (paragraph)}




%\begin{tabbing}
%	\hspace*{.25in} \= \hspace*{.25in} \= \hspace*{.25in} \= \hspace*{.25in} \= \hspace*{.25in} \=\kill
%	{\bf function} find (A,x,n)\\
%	\> j $\leftarrow$ 0\\
%	\> {\bf while} j < n\\
%	\>\> {\bf if} A[j]=x\\
%	\>\>\>  {\bf return} j  \\
%	\>\> j $\leftarrow$ j+1\\
%	\> {\bf return} -1
%\end{tabbing}


%\begin{figure}[htbp!]
%		\centering
%		\includegraphics[width=0.6\textwidth]{lec26.png}
%		\caption{Linked List}%\label{book}
%		\vspace{-1em}
%\end{figure}






%\begin{align}
%A = 
%\begin{bmatrix}
%A_{11} & A_{21} \\
%A_{21} & A_{22}
%\end{bmatrix}
%\end{align}






